\documentclass[a4paper,11pt]{article}

\usepackage[utf8]{inputenc}
\usepackage[czech]{babel}
\usepackage[left=2cm,top=3cm,text={17cm,24cm}]{geometry}
\usepackage{graphicx}
\usepackage{listings}
\usepackage{url}
\usepackage{amsmath, amsthm, amssymb}

\begin{document}

\begin{center}
\textsc{\Huge Fakulta informačních technologií\\
Vysoké učení technické v~Brně\\}

\LARGE \title{Algoritmy\\
\textbf{4. Nejkratší cesta v grafu\\}
Náhradní projekt\\}


\hfill \author{Tučková Martina xtucko00@stud.fit.vutbr.cz\\
 Tussupová Kymbat xtussu00@stud.fit.vutbr.cz\\
 Lázňovský David xlazno00@stud.fit.vutbr.cz\\
 Drahník Lukáš xdrahn00@stud.fit.vutbr.cz}
\end{center}


{\let\newpage\relax\maketitle}

\newpage

\section*{Obsah}
\begin{itemize}
  \item Úvod
  \item Použité algoritmy
  \begin{enumerate}
  \item Bellman-Fordův algoritmus
  \item Dijkstrův algoritmus
  \item Floyd-Warshallův algoritmus
  \end{enumerate}
  \item Závěr
  \begin{enumerate}
  \item Příklad použití
  \item Omezení programu
  \item Metriky
  \item Použitá literatura
  \end{enumerate}
\end{itemize}

\newpage

\section*{Úvod}
Tato dokumentace popisuje nalezení nejkratší cesty v grafu pomocí implementace známých algoritmů do předmětu Algoritmy.  Tyto budou popsány samostatně v jednotlivých kapitolách.

\section{Použité algoritmy}
\subsection{Bellman-Fordův algoritmus}
Bellman-Fordův algoritmus slouží pro hledání nejkratších cest v ohodnoceném grafu, kde mohou být záporně ohodnoceny hrany, ale není obsažen cyklus záporné délky – v tomto případě je algoritmus schopen jeho detekce, a to buď pomocí čítače, který ukončí výpočet po překročení maximálního možného počtu iterací algoritmu, anebo přidáním jednoho vnořeného cyklu.
\par
Hlavní operací algoritmu je relaxace hran, kde hrají roli dva uzly a mezi nimi vedoucí hrana. Pokud je vzdálenost mezi uzly u a v kratší než dosavadně spočítaná, tak ji přepočteme a opravíme předchůdce uzlu v na u. Tato metoda je využívána i u Dijkstrova algoritmu.
\par
Asymptotická složitost je $O(V*E)$.

\subsection{Dijkstrův algoritmus}
Tento algoritmus byl vymyšlen nizozemským informatikem Edsgerem Dijkstrou v roce 1959. Soustředí se na prohledávání grafu do šířky , kde se vlna šíří v závislosti na vzdálenosti od zdroje.
\par
Jeho použití je omezeno na kladné hrany – při záporných není garance nejkratší možné cesty a je výhodnější použít jiné algoritmy; dále je potřeba, aby graf byl ohodnocený – neohodnocený lze však snadno převést. Vstupem by tedy měl být nezáporně ohodnocený graf a počáteční uzel.
\par
Lze jej naimplementovat podle prioritní fronty, tedy s asymptotickou časovou složitostí $O(|V|^2+|E|)$ nebo pomocí binární $(O((|E|+|V|)log|V|))$ či Fibonacciho haldy $(O(|E|+|V|log|V|))$.

\subsection{Floyd-Warshallův algoritmus}
Tento algoritmus si zakládá především na implementační jednoduchosti a je příkladem dynamického programování. Poradí si se záporně ohodnocenými hranami, ale ne se záporným cyklem - proto se pro správnost funkčnosti algoritmu předpokládá, že jej graf neobsahuje.
\par
Jeho základem je operace s maticí sousednosti, kterou je ovšem nutno vytvořit. Na diagonále této matice jsou samé nuly, protože vzdálenost z uzlu do stejného uzlu je nula. Na indexu (i,j) je hodnota délky hrany mezi dvěma uzly (i,j), v ostatních indexech je nekonečno. Jeho použití se vyplatí v hustých grafech, ale naopak v řídkých grafech s nezáporně ohodnocenými hranami je výhodnější použít Dijkstrův algoritmus.
\par
Asymptotická časová složitost je $O(|V|^3)$.

\section{Závěr}
Projekt byl zajímavý a naučili jsme se díky němu používat vhodné algoritmy a poznávat jejich výhody a nevýhody. Potřebná látka byla probírána okrajově až k termínu odevzdání projektu, ale nebyl problém si vše zjistit sami a vypracovat zadání díky samostudiu.

\newpage

\subsection{Příklad použití}

Program umožňuje zobrazit nápovědu spuštěním programu s option -h. Pomocí options lze zadat jakýkoliv typ grafu. Vylučuje se použití non-rated [-u] s rated [-r]. Pokud není uvedeno orientovaný graf, bude s grafem zacházeno jako s neorientovaným, u rated / non-rated dojde ke zjištění automaticky - tzn. pokud graf bude obsahovat pro všechny hrany stejnou hodnotu nebo žádnou, jedná se o neohodnocený.

\lstset{language=Bash}
\begin{lstlisting}[frame=single,breaklines]
  ./shreya -i tests/test06 A D 2> /dev/null

  Dijkstra:


  -> A -> B -> D
  -> A -> C -> D

\end{lstlisting}

\lstset{language=Bash}
\begin{lstlisting}[frame=single,breaklines]
  ./shreya -i tests/test07 A B -o -r 2> /dev/null
  Floyd Warshall:

  -> A -> C -> F -> B

  Bellman Ford:

  -> A -> C -> F -> E -> B
  -> A -> C -> F -> B
  -> A -> C -> E -> B
  -> A -> D -> F -> E -> B
  -> A -> D -> F -> B
\end{lstlisting}

\subsection{Omezení programu}
Název vrcholu nesmí být číslo. Floyd-Warshallův algoritmus nachází pouze jednu z více nejkratších cest (pokud více existuje) a to podle pořadí hran zadaného grafu na vstupu.

\subsection{Metriky}

Kód programu je rozdělený do modulů. Každý modul obsahuje hlavičkový soubor s definicemi metod. \newline\newline
Počet souborů: 15 \newline
Počet řádků zdrojového kódu: 1211

\subsection{Použitá literatura}

\nocite{*}

%% BIBLIOGRAPHY
\bibliography{local}
\bibliographystyle{plain}

\newpage
\thispagestyle{empty}

\end{document}
%% END OF FILE
